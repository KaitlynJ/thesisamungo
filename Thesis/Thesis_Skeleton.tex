% This is the Reed College LaTeX thesis template. Most of the work
% template. Later comments etc. by Ben Salzberg (BTS). Additional
% restructuring and APA support by Jess Youngberg (JY).
% Your comments and suggestions are more than welcome; please email
% them to cus@reed.edu
%
% See http://web.reed.edu/cis/help/latex.html for help. There are a
% great bunch of help pages there, with notes on
% getting started, bibtex, etc. Go there and read it if you're not
% already familiar with LaTeX.
%
% Any line that starts with a percent symbol is a comment.
% They won't show up in the document, and are useful for notes
% to yourself and explaining commands.
% Commenting also removes a line from the document;
% very handy for troubleshooting problems. -BTS

% As far as I know, this follows the requirements laid out in
% the 2002-2003 Senior Handbook. Ask a librarian to check the
% document before binding. -SN

%%
%% Preamble
%%
% \documentclass{<something>} must begin each LaTeX document
\documentclass[12pt,twoside]{reedthesis}
% Packages are extensions to the basic LaTeX functions. Whatever you
% want to typeset, there is probably a package out there for it.
% Chemistry (chemtex), screenplays, you name it.
% Check out CTAN to see: http://www.ctan.org/
%%
\usepackage{graphicx,latexsym}
\usepackage{amsmath}
\usepackage{amssymb,amsthm}
\usepackage{longtable,booktabs,setspace}
\usepackage{chemarr} %% Useful for one reaction arrow, useless if you're not a chem major
\usepackage{rotating}

% Modified by CII
\usepackage[hyphens]{url}
\usepackage{hyperref}
\usepackage{lmodern}

% Added by CII (Thanks, Hadley!)
% Use ref for internal links
\renewcommand{\hyperref}[2][???]{\autoref{#1}}
\def\chapterautorefname{Chapter}
\def\sectionautorefname{Section}
\def\subsectionautorefname{Subsection}

\usepackage{caption}
\captionsetup{width=5in}

% \usepackage{times} % other fonts are available like times, bookman, charter, palatino

\title{Patients as Consumers: the cohaptation and consumption of health at
urgent care centers}
\author{Kaitlyn R. Jackson}
% The month and year that you submit your FINAL draft TO THE LIBRARY (May or December)
\date{Fall 2015}
\division{History and Social Sciences}
\advisor{Jessica F. Epstein}
%If you have two advisors for some reason, you can use the following
%\altadvisor{Your Other Advisor}
%%% Remember to use the correct department!
\department{Sociology}
% if you're writing a thesis in an interdisciplinary major,
% uncomment the line below and change the text as appropriate.
% check the Senior Handbook if unsure.
%\thedivisionof{The Established Interdisciplinary Committee for}
% if you want the approval page to say "Approved for the Committee",
% uncomment the next line
%\approvedforthe{Committee}

% Below added by CII

%%% Copied from knitr
%% maxwidth is the original width if it's less than linewidth
%% otherwise use linewidth (to make sure the graphics do not exceed the margin)
\makeatletter
\def\maxwidth{ %
  \ifdim\Gin@nat@width>\linewidth
    \linewidth
  \else
    \Gin@nat@width
  \fi
}
\makeatother

\renewcommand{\contentsname}{Table of Contents}

\setlength{\parskip}{0pt}

\providecommand{\tightlist}{%
  \setlength{\itemsep}{0pt}\setlength{\parskip}{0pt}}

\Acknowledgements{

}

\Dedication{

}

\Preface{

}

\Abstract{

}

\usepackage{tikz} \usepackage{setspace}

%%
%% End Preamble
%%
%

\begin{document}

      \maketitle
  
  \frontmatter % this stuff will be roman-numbered
  \pagestyle{empty} % this removes page numbers from the frontmatter

  
  
  % Add table of abbreviations?

      \hypersetup{linkcolor=black}
    \setcounter{tocdepth}{2}
    \tableofcontents
  
  
  
  
  
  \mainmatter % here the regular arabic numbering starts
  \pagestyle{fancyplain} % turns page numbering back on

  \chapter*{Introduction}\label{introduction}
  \addcontentsline{toc}{chapter}{Introduction}
  
  \onehalfspacing
  
  The past two decades have seen a surge in a new form of medical
  practice: walk in clinics designed with an emphasis on acute, quick and
  cheap care. WHile they went through a phase of many names, most people
  in America are familiar and personally acquainted with urgent care
  centers. These urgent care centers focus on acute episodic care with a
  substantial emphasis on customer service. There is no official
  definition of what constitutes an urgent care center, but the scope of
  services provided generally falls between that of a primary care
  doctor's office and an emergency department.
  
  The first of these centers opened in the United States in the early
  1980's., with no more than a handful in operation at the time.
  Unfortunately, (at least as far as early investors were concerned), the
  industry rapidly declined, and the few clinics which had opened were
  largely obsorbed into larger hospitals and healthcare groups. Ten years
  later, in the mid-1990's , the industry again began growing rapidly,
  quickly growing to between 12,000 and 20,000 centers today. By the
  UCAOA's estimate, in 2014 approximately two new urgent care centers were
  opening in the United States each week.
  
  Such recent and rapid expansion of the industry has been heavily
  examined by the media, which often attribute growth to a diverse set of
  market factors such as long wait times for primary care appointments,
  crowded emergency departments and patient demand for more accessible
  care, including after-hours appointments (Yee, Lechner, and Boukus
  2013). Yet despite the rapid development of the industry and the great
  interest sociologists have historically taken in America's health care
  system, hardly any scholarly research has been done on why these centers
  are coming to play a major part of the healthcare system or what their
  patterns of use are. While many are quick to point towards long wait
  times, and the difficulty of finding doctors in the current healthcare
  system, the repurcussions of the new turn torwards urgent care centers.
  
  In the following analysis, I will attempt to situate the rise of such
  clinics within the existing sociological research on the American
  healthcare system, generating hypotheses about why patients are turning
  towards such centers and away from primary care and/or emergency
  department use. In Chapter one, I highlight the rise in the number of
  urgent care centers in the United States since the early 1990's, showing
  the rising prominence of these institutions. I also provide background
  on the services and care these faciilities offer, focusing on how they
  are advertised and their own obvious institutional goals. I situate
  urgent care centers into a wider context of the American healthcare
  system's varied actors and organizations, and explore the ways these
  insitutions might offer a response to common complaints regarding the
  accessibility and organization of the more established healthcare
  faciliites.
  
  Chapter two turns to the current sociological understandings of the
  healthcare system, and examines how the competing theories of the
  patient/doctor relationship work when examining these actors in an
  urgent care setting, focusing on theories of consumerism in modern
  healthcare. I highlight how these new organizations might informally
  facilitate an arm's length relationship between practitioner and
  patient, eliminating the close ties many medical sociologists have
  identified as vital to an optimal doctor patient relationship. In this
  chapter I also highlight the current gaps in the literature surrounding
  new forms of medical care facilities which have been largely ignored by
  sociologist and other scholars attempts to understand health care in
  America.
  
  Chapter three offers a description of the National Ambulatory Medical
  Care Survey where I am drawing my anlysis from, and including a
  description of the methods I use to examine patient visits to urgent
  care. These include both a k-modal hieracrchal clustering exploratory
  analysis used to identify groups of similar patients using urgent care
  and a logistic analysis of those choosing to use urgent care as their
  primary care facility. Chapters four and five present these analyses,
  and discuss the findings in light of the research.
  
  ** Insert thesis statment. ``I suggest\ldots{}'' **
  
  In the conclusion, I explore the implications of the tensions between
  what is thought important to the doctor-patient relationship, and the
  reality of accessing healthcare in America today. I argue that\ldots{}
  ** really need to clarify what I'm arguing here **. I conclude with
  implications and suggestions for ways in which urgent care centers could
  be incorporated into the traditional healthcare model.
  
  \chapter{Background Information}\label{background-information}
  
  \onehalfspacing
  
  \chapter{Literature Review}\label{literature-review}
  
  \onehalfspacing
  
  ** Need intro to this\ldots{} make more cohesive **
  
  \section{Moving away from Primary
  Care}\label{moving-away-from-primary-care}
  
  The shift towards urgent care centers will undoubtedly have profound
  consequences for the patient-practitioner relationship and the patient's
  role in the medical care system, but so far these consequences remain
  largely unknown. Fortunately, sociologists have had a long-standing
  concern with such professional-client interactions, particularly within
  medicine (Freidson 1961, Bloom 1963, Mechanic 1968), and in order to
  better understand what effects the shift away from primary care may have
  on patients, one can look to a large body of research on how
  institutional and organizational environments directly affect a
  patient's experience in health care.
  
  A major sociological theory of the doctor-patient relationship begins by
  speculating that the bureaucratization of modern social institutions has
  had drastic consequences on the medical profession: as opportunities for
  close personal contacts diminish, problems which were originally handled
  in familial, social, and religious contexts are transferred to `formal
  sustaining practitioners' (Mechanic 1966). In such societies, the
  prescribed structure of the doctor patient relationship then provides a
  legitimation for the expression of intimacy and the request for help,
  offering an explanation as to why sociologists and anthropologists of
  medicine have long observed the variance and social nature of health
  problems brought to a physician (The Doctor, His Patient, and the
  Illness).
  
  These same theorists locate the stability of this doctor-patient
  relationship in the fact that the physician acts as the patient's agent,
  yet one can immediately recognize that urgent care centers may not be
  equipped to facilitate this relationship in the same way that the
  traditional primary care practice is seen to (Lupton 1997). The premise
  of urgency in such practices, the quickness with which patients are
  seen, and the targeted focus on acute problems all serve to create
  considerable doubt towards the ability of physicians within such an
  organizational context to fulfill the sociological role thought so
  important in previous literature.
  
  Additionally, those who study health services have recognized that a
  patient's medical history is a primary source of information regarding
  treatment, and primary care practitioners have been known to draw upon
  this as a valuable resource (Draper and Smits 1975; Miller et al. 2010).
  The Millis report, commissioned by the American Medical Association to
  review the current status of physicians, recalls the archetype of the
  medical professional:
  
  ``The general practitioner of revered memory knew his patients\ldots{}
  and provided continuing care through the course of minor ailments and
  majors emergencies. His deficiencies\ldots{} were partly offset by
  intimate knowledge of his patients, the support he gave them, and the
  trust and confidence his services engendered.''
  
  Urgent care centers however have no emphasis on maintaining
  patient-doctor ties and there is no reliable system to ensure a patient
  sees the same doctor even if they have been there before.
  
  Yet while this may seem problematic, a large body of recent sociological
  literature belongs to a growing number of scholars who are challenging
  the importance, and even relevance, of the traditional primary care
  physician in modern medicine. Those who have been observing developments
  in the patient-physician relationship over the past 30 years argue that
  the last quarter of the 20th century saw a dramatic reconfiguration of
  society, especially in regards to health services, which has had
  profound effects on an individual's relationship to the healthcare
  system. So while some were witnessing what was, for Starr (1982), ``the
  social transformation of medicine,'' and for many, ``the end of the
  golden age of doctoring'' (McKinlay \& Marceau 2002).
  
  \subsection{The end of Medical Professional Dominance in the
  US?}\label{the-end-of-medical-professional-dominance-in-the-us}
  
  Almost since sociologists first became interested in the medical
  profession, the sociology of medicine was deeply joined with studies of
  professionalism. In the U.S., Doctors were seen as the paragon of the
  `professional': respected, organized, in control, and above all else,
  firmly established in their positions. Such traits led many to study
  what effects this had on medical care and how such professional
  dominance of medical practitioners shaped what services were provided.
  Many found that such dominance allowed the profession to block off areas
  of study entirely, or to ignore diseases they did not want to pursue.
  Such ``modern doctors'' worked within a ``sovereign profession'' (Starr
  1982), serenely dispensing both medical care and authoritative judgment.
  Freidson (1988, p.384) comments that before 1970's, U.S. medicine ``was
  at a historically unprecedented peak of prestige, prosperity and
  political and cultural influence---perhaps as autonomous as it is
  possible for a profession to be.''
  
  When one thinks of the healthcare industry today, it is hard to call to
  mind such professional cohesion. While doctors remain one of the more
  highly respected career choices in America, their prestige has certainly
  dropped since the 20th century (Heritage and Maynard 2006). The term
  doctor is now applicable to a wide variety of sub-professions and
  specialties and the medical profession has become extremely diversified.
  Along with such specialization, big changes to Medicare and Medicaid
  legislation and the growth of third-party payers and for-profit medical
  service corporations created conditions which further removed the doctor
  from their traditional roles, eroding the political and cultural
  influence of the profession and threatening the cultural authority and
  technical autonomy of medicine (Starr 1982, Freidson 1988). It should be
  noted that such changes may not have necessitated a loss of professional
  dominance. In fact, with the expansive growth on spending in the
  healthcare sector, it is possible that doctors could have further
  cemented their professional medical authority. But most scholars have
  observed that the opposite has happened, and instead many view the past
  30 years as the end of the authoritative medical professional.
  
  With such a history of professional sominence, many have examined the
  observed decline in depth, and the change is largely seen as a
  consequence of a loss of trust with the medical profession as a whole
  that began some time around the late 1970's (Timmermans and Oh 2010).
  During the `golden age', public surveys reported extremely high levels
  of trust in physicians, however this declined from 72 percent in 1966 to
  37 percent in 1981 (Lipset and Schneider 1982). With high levels of
  doubt towards medical professionals, suspicion grew about physicians
  acting in patients' best interests (Reeder 1972) and patients began to
  question the validity of the medical doctor as an authoritative figure
  of biological truth.
  
  These shifts in the medical profession occurred parallel to changing
  norms surrounding the role of the patient in their own health care, and
  the last 30 years of scholarly research have seen a reconfiguration of
  the patient from passive recipient of care from their doctor to a
  critical consumer of health services (Barker 2008; Lupton, Donaldson,
  and Lloyd 1991; Timmermans and Oh 2010). Theories of medical consumerism
  developed from economists studying the healthcare sector, and they begin
  with a similar hypothesis as the economic rational choice model,
  assuming that patients act as rational actors in the context of a
  medical encounter (Timmermans and Oh 2010). In other words, individuals
  act in a calculated manner to engage in self-improvement or health, and
  they are generally skeptical about expert knowledge (Lupton 1997).
  According to the literature, this trend began to express itself in the
  form of solicitation of second opinions and a sense of
  interchangeability of medical practitioners during the 1980's, just as
  distrust of the industry reached its peak (Gray 1997). The idea that
  patients could shop around and compare services and prices was heavily
  popularized, and patients increasingly began to make autonomous
  decisions when selecting physicians (Hibbard and Weeks 1987, Lupton
  1991).
  
  Urgent care centers fit neatly into such a conceptualization of health
  services, and thus offer a key area of analysis in better understanding
  the developing roles of the consumer-patient within the larger
  healthcare industry. An important aspect of the research on the
  developing patient-consumers emphasizes the expansion of bargaining
  power on the part of the patient that came with the shift (Reeder 1972).
  A patient may now shop around the marketplace of health care, and that
  many are now choosing urgent care centers is undeniable given the
  industry's rapid expansion.
  
  \subsection{What does this mean for health
  care?}\label{what-does-this-mean-for-health-care}
  
  In light of such research, sociologists and those who study at the
  intersection of health and social behaviors have begun to re-examine the
  importance of a close relationship between doctor and patient in an
  attempt to respond to the consumer-patient model, developing a growing
  body of research which seeks to define the most effective components of
  patient-physician interactions and to reaffirm the place of the primary
  care physician in modern medicine. Common to most of these studies are
  the elements of `trust, compassion, communication, and clinical
  competence' (Heritage and Maynard 2006; Phillips and Bazemore 2010,
  others at bottom). As an example, a 2002 study on clinical outcomes for
  low income women over the age of forty found that women who rated
  highest their doctor's ability to take care of all of their health care
  needs had 11 times the odds of `trusting their physician' and 6 times
  the odds of finding their physicians `compassionate and communicative',
  compared to those with the lowest level of comprehensiveness (O'Malley
  and Forrest 2002).
  
  With such knowledge that close ties between physicians and patients have
  a direct impact on the perceived quality of care, the importance of
  examining the rise of urgent care centers become obvious. If patients
  are acting as rational consumers, which of them are choosing to receive
  their care outside of a primary care office, and what consequences does
  this have for their medical outcomes? The lack of research into the
  characteristics of urgent care patients make it difficult to answer such
  questions. Indeed, as of now, there has been no academic attempt to
  place the rapidly growing industry within the sociology of medicine, nor
  does there even exist an official definition of what constitutes an
  urgent care center.
  
  \section{Urgent Care Centers as an extension of the Welfare
  State}\label{urgent-care-centers-as-an-extension-of-the-welfare-state}
  
  The prevailing understanding of urgent care centers interprets the
  industry's growth as a development that grew out of emergency department
  overcrowding that began during the financial tightening of the 1980's.
  Thus, each clinic is often conceptualized as a smaller instance of a
  traditional hospital's triage center, supplying many services which one
  could receive at an emergency department (Anon n.d.; Mehrotra et al.
  2008; Rubin 2012). One study which compared emergency department
  services with urgent care centers found that, for all but the most
  extreme care and emergent care needs, urgent care centers could handle
  almost all of emergency department traffic and that their services were
  remarkably analogous (Anon n.d.). Similarly, the unique features of
  emergency departments which set them apart from more traditional avenues
  of care such as the promise to be seen regardless of insurance coverage
  and the flexible hours are mirrored in most urgent care centers (Weinick
  and Betancourt n.d.).
  
  Given these similarities, urgent care could in many ways be considered a
  response by the healthcare industry to emergency department overuse,
  which continue to struggle underneath a lack of resources. Suitably, the
  body of sociological research on emergency departments can be used to
  better comprehend how urgent care centers are impacting the allocation
  of healthcare. Many medical sociologists have analyzed the
  public/private divide in the healthcare industry, emphasizing the
  unequal quality of care received depending upon which type of health
  service you utilize (Dutton 1978; Luftey and Freese 2005). In response
  to these inequalities in the private sector, the emergency room has long
  been understood as a response to a stratified system, and it is largely
  considered a social welfare institution (Gordon 1999). ``The hospital
  emergency department is perhaps the only local institution where
  professional help is mandated by law, with guaranteed availability for
  all persons, all the time, regardless of the problem'' (Ullman, Block,
  and Stratmann n.d.), and was for a long time widely considered one of
  the few access points into the healthcare system for very low income
  individuals.
  
  Accordingly, because urgent care centers offer a similar alternative to
  primary care for those who cannot procure a conventional family
  practitioner, they can be considered an extension of a much needed
  welfare institution necessary to circumvent an extremely stratified
  healthcare industry. This becomes more credible when trends in urgent
  care centers are examined in conjunction with emergency departments. By
  all indications, the demand for emergency departments far outweighs
  their capacity, and this is only expected to grow in the future (Weinick
  2010). Around the time that urgent care centers began to emerge in the
  U.S., emergency departments were seeing record breaking levels of
  non-acute visits, resulting in severe overcrowding and shortages
  (Shortcliffe). These numbers have only risen in the past 20 years,
  especially in urban areas with large populations of low-income residents
  (Anon n.d.). Not only does this rise in demand correspond to the growth
  of the urgent care industry, one economic analysis which took cities
  with comparably high levels of non-emergent ED usage and examined the
  effect of a growing number of urgent care centers found that cities
  which have seen a large growth in these facilities have statistically
  lower overcrowding in emergency departments (O'Malley 2013).
  
  Thus, urgent care centers usage would be expected to closely resemble
  that of the emergency department, especially regarding non-emergent
  care. This leads to the first hypothesis which will be tested by current
  data on urgent care center usage. Emergency department usage has been
  studied by many medical professions and sociologists, with a particular
  focus on the ways in which it serves as a welfare institution, and a few
  central patterns will be used here in the comparison with urgent care
  centers. Primarily, the low average income of patients and high levels
  of uninsured visits to emergency department patients have been a steady
  trend in the last 30 years (Ullman et al. n.d.). Patterns of use for
  emergency departments also show low levels of `returns' (individuals who
  come back with the same problem) and higher traffic during hours when
  normal primary care physician offices are closed, such as late at night
  and on the weekend (Anon n.d.; O'Malley 2013; Ullman et al. n.d.).
  Accordingly, if like emergency departments, urgent care centers act as a
  a welfare institution for those who cannot procure a traditional primary
  care physician, we would expect to see similar trends as have been
  observed in hospital emergency departments. These are summarized in
  Hypothesis 1 below.
  
  \paragraph{Hypothesis 1: Urgent care centers will experience high levels
  of visits from uninsured and low SES patients, have low rates of second
  visits, and higher rates of visits during non-traditional hours
  (weekends).}\label{hypothesis-1-urgent-care-centers-will-experience-high-levels-of-visits-from-uninsured-and-low-ses-patients-have-low-rates-of-second-visits-and-higher-rates-of-visits-during-non-traditional-hours-weekends.}
  
  \section{Urgent Care Centers as a New Model of Primary
  Care}\label{urgent-care-centers-as-a-new-model-of-primary-care}
  
  Yet another understanding of urgent care centers arises when one
  considers the larger institutional context within which the overuse of
  emergency departments and the boom in the urgent care industry occurred.
  According to medical and organizational sociologists, the institutional
  narrative of the American healthcare industry since the 1980's is one of
  privatization and the transfer of power from professionals and the
  government towards the private sector and market control (Waitzkin
  2000). Urgent Care centers thus fall into a rapidly expanding category
  of new, privately funded modes of healthcare services---other examples
  being retail clinics, private hospitals and home care
  organizations---which are provided and predominantly paid for by private
  actors (Anon n.d.).
  
  This framework has yet to be fully examined by scholars, but a few
  studies which have attempted to examine the place of urgent care centers
  in new healthcare markets have pointed towards such change, highlighting
  the privatization phenomenon as a possible explanation in the rise in
  numbers of such clinics over the last two decades or so (Rubin 2012;
  Weinick, Bristol, and DesRoches 2009; Yee et al. 2013). Shortages in
  public hospital staffing and facilities and the rising cost of care in
  the U.S. are seen as having created both demand and an opening for a new
  market within healthcare, which has been filled by new forms of medical
  service (Weinick and Betancourt n.d.). And while the trend towards
  privatization has been seen at times as both a symptom of the loss of
  professional dominance by medical practitioners and the cause of it, the
  similarities in organizational structure to primary care practices cast
  doubt on the hypothesis that these new forms of care are simply
  emergency department overflow. As one analysis observed: ``while urgent
  care reflects some similarities to emergency departments, we find that
  in other areas -- most notably reimbursements, primary payer
  distribution, and physicians' salaries -- urgent care centers seem far
  more similar to office-based family medicine practices'' (Weinick and
  Bristol 2008).
  
  Such observations lead one to a different conclusion about urgent care
  than those who liken the clinics to smaller triage centers created to
  handle emergency department overflow. Instead, urgent care centers can
  be conceptualized as a move by the healthcare industry towards deeper
  privatization, and possibly a response to medical professional authority
  in jeopardy. When medical sociologists first began pointing to ``the end
  of the golden era of doctoring,'' Stefan Timmermans responded by
  pointing to the long history of adaptability by the medical profession,
  which has managed to transform itself before in the wake of
  institutional change many times before (Timmermans, Whooley).
  
  Consequently, if the privatization of healthcare proves to be the
  primary explanation of why urgent care centers are beginning to dominate
  the acute-care market, one would expect to see trends in use mirror
  those of traditional primary care physicians rather than emergency
  departments. Historically, primary care often acted as a first contact
  point for insured patients for any acute, non-emergent health concerns
  of these individuals (Jost 2003). While many primary care offices will
  accept at least a small number of Medicaid and Medicare payments, they
  often have a large patient base of privately insured, financially
  well-off individuals (Cunningham et al. 1999). Such offices are often
  only open during a standard work week's hours, a facet often noted as
  functioning to limit access for those who cannot take off work to go to
  the doctor. Lastly, primary care is often set apart from other forms of
  care due to the relationship and medical history that develops between
  the doctor and patient over years of care exchange (Miller et al. 2010).
  Primary care physicians thus emphasize holistic view of health care, and
  for those patients that do have a primary doctor, they are encouraged to
  return to the same clinic. If urgent care facilities are to be
  understood as a new face on a conventional sector of the health services
  industry, we should expect to see this last factor of primary care
  present in those going to urgent care centers. These usage trends are
  summarized in Hypothesis 2 below:
  
  \paragraph{Hypothesis 2: Urgent care centers will experience high levels
  of visits from insured patients across SES statuses, have high rates of
  second visits, and will not have significantly higher rates of visits on
  weekends.}\label{hypothesis-2-urgent-care-centers-will-experience-high-levels-of-visits-from-insured-patients-across-ses-statuses-have-high-rates-of-second-visits-and-will-not-have-significantly-higher-rates-of-visits-on-weekends.}
  
  \section{Urgent Care Centers: Emergency Rooms for the
  Wealthy?}\label{urgent-care-centers-emergency-rooms-for-the-wealthy}
  
  An alternative explanation for the rise of urgent care use combines both
  of the previous theories by conceptualizing the urgent care center as an
  occurrence of boundary work by upper middle class and wealthy Americans
  who have been pushed out of emergency departments by low SES patients.
  Sociologists have long been interested in the ways that class can be
  defined and reconstituted through boundary work (Pachucki, Pendergrass,
  and Lamont 2007). Research on symbolic boundaries---the conceptual
  distinctions made by social actors in categorizing people, practices,
  tastes, attitudes and manners---and their interactions with more durable
  and institutionalized social differences such as class and race has
  shown that individuals often use methods of exclusion through
  organizational settings in order to solidify social boundaries Zietsma
  and Lawrence 2010.
  
  Thus, if it is true that urgent care centers are largely operationally
  analogous to emergency departments as is often observed, it could be
  that such organizations serve as an alternative for wealthy individuals
  who seek to socially distance themselves from a space largely inhabited
  by low SES patients. Such a hypothesis would explain why the industry's
  rapid growth coincides with a remarkable strain on emergency
  departments' capacities due to an influx of low SES patients. This
  explanation also accounts for the fact that urgent care centers are
  often more expensive due to their pay-per-service model than emergency
  departments (Anon n.d., Anon n.d.; Ullman et al. n.d., n.d.). Lastly,
  this theory offers an explanation as to why urgent care centers have
  developed along side of, and with many of the same services as,
  emergency departments Weinick et al. 2009.
  
  If this hypothesis were the case, one would expect to observe usage
  trends much like Hypothesis 1, but with key differences. Primarily,
  regardless of insurance coverage, we should expect to see little to no
  low-SES users in such clinics. We would also expect to see patterns of
  use by the wealthy to mimic their would-be use of emergency departments,
  such as high volumes of visits during times when their primary care
  physicians are closed, emergency and injury related visits and almost no
  return visits.
  
  \paragraph{Hypothesis 3: Urgent care centers will experience high levels
  of visits from high SES patients, have low rates of second visits and
  will have large proportions of visits which qualify as acute and
  emergent care during non-primary care
  hours.}\label{hypothesis-3-urgent-care-centers-will-experience-high-levels-of-visits-from-high-ses-patients-have-low-rates-of-second-visits-and-will-have-large-proportions-of-visits-which-qualify-as-acute-and-emergent-care-during-non-primary-care-hours.}
  
  If such trends were observed, the urgent care center could then be
  understood as anlaogous to an emergency department for the wealthy.
  
  \onehalfspacing
  
  \begin{verbatim}
  Warning: package 'knitr' was built under R version 3.2.3
  \end{verbatim}
  
  \chapter{Methods}\label{methods}
  
  In this thesis I am attempting to accomplish three distinct goals: to
  examine the current health-seeking trends in America, something that has
  yet to have been done in medical sociology with urgent care in
  consideration; then, to examine the question of \textit{whom}: by
  clustering the groups who have chosen to go to urgent care; and finally,
  to use the quantitative data and sociological theories to hypothesize on
  the reasons behind the decisions.
  
  Using National Ambulatory Medical Care Survey data for the years
  2008-2012, I utilize unsupervised machine learning methods in order to
  develop a descriptive categories of urgetn care center patients. From
  the exploratory anlaysis, I then test the resulting variables in a
  logarithmic regression analysis, devopling a model of urgent care
  seekers.
  
  The initial exploratory nature of the analysis is motivated by the
  current lack of statistical analysis and social theory surround urgent
  care centers, despite their rapid progression as an acceptable
  replacement for primary care (erhm). The results of the findings will
  hopefuly allow those intent on locating urgent care centers within the
  larger contextual framework of the American health care system to draw
  on the revealed typologies of urgent care seekers in order to better
  understand the industry's rapid growth.
  
  \section{Data}\label{data}
  
  Empirical exploration of the theories proposed in chapter 1 require data
  that provides an abundance of variables which may or may not be
  statistically important but which we cannot initally rule out, as well
  as a large size since the phenomenon is still comparatively rare when
  talking about how patients access primary care in the US. I thus chose
  the National Ambulatory Medical Care Survey (NAMCS), which is a national
  survey designed to provide researchers in the medical and social science
  fields ``accurate and reliable information about the provision and use
  of ambulatory medical care services in the United States''.
  
  Ambulatory care is defined by the survey as health services or acute
  care services provided to patients on an outpatinet basis, without an
  overnight stay, and every year NAMCS surveys visits to non-federal
  employed office-based physicians are collected from a representative
  sample of the United States. These surveys contain information about how
  the patients utilize physician services and hospital outpatient and
  emergency department services, the conditions most often treated, and
  the diagnostic and therapeutic services rendered, including medications
  prescribed. This data served the purpose of the current study because it
  is both representative of the larger trends in the United States and
  includes specific information regarding urgent care centers which can be
  used to explore that particular American health care trend in
  particular.
  
  I specifically examined the group of visits were coded as having been at
  ``Urgent Care Centers/Freestanding clinics'' by the NAMCS. While the
  combined years produced a dataset of 123,123 observations, only 3,863 of
  those occurred at urgent care centers (about 3 percent). Of these
  visits, I limited the analysis to patients over the age of 18, bringing
  the sample to 3,224 visits to urgent care centers.
  
  Some limitations to the data should be noted. There may be related
  errors given that as the popularity of urgetn care centers have risen,
  so too have the number that participated in the NAMCS. In 2008, there
  were 842 visits surveyed compared to 1168 surveyed in 2010.
  
  \section{Variable Selection and Summary
  Statistics}\label{variable-selection-and-summary-statistics}
  
  The variables I chose to include in my analysis were chosen with both
  data availaility and sociological theory in mind. Initially I am
  interested in looking at demographics surrounding age, race, gender, and
  socio economic status. Inital clusters included dummy variables for
  self-pay, private insurance, race (white, black, hispanic, asian, 2+),
  sex, rural, and wealthy. To identify the subsets, I took random samples
  from the cases which were recording as having been at urgent care
  centers using the dplyr package sample\_n, which allows for a random
  campling of rows from a table. from the cases which were recording as
  having been at urgent care centers. For the three years in question a
  total of 2459 visits to urgent care were surveyed. In order to get
  workable training data for the unsupervised clustering, I set aside a
  random sample of 250 visits for test data, and proceeded to randomly
  sample from the 2209 cases available, running the analysis on 250
  observations at a time.
  
  \newpage
  
  \singlespacing
  
  \begin{longtable}[c]{@{}lll@{}}
  \caption{Summary Statistics 1 \label{tab:sums}}\tabularnewline
  \toprule
  Variable & Category & Percentage\tabularnewline
  \midrule
  \endfirsthead
  \toprule
  Variable & Category & Percentage\tabularnewline
  \midrule
  \endhead
  Sex & &\tabularnewline
  & Female & 57.1\tabularnewline
  & Male & 42.9\tabularnewline
  AgeGroup & &\tabularnewline
  & 15-24 years & 8.68\tabularnewline
  & 25-44 years & 22.89\tabularnewline
  & 45-64 years & 29.63\tabularnewline
  & 65-74 years & 13.93\tabularnewline
  & 75 years and over & 12.58\tabularnewline
  & Under 15 years & 12.3\tabularnewline
  Race & &\tabularnewline
  & Black & 9.15\tabularnewline
  & Other & 3.87\tabularnewline
  & White & 86.98\tabularnewline
  PaymentType & &\tabularnewline
  & All sources of payment are blank & 0.44\tabularnewline
  & Medicaid & 10.67\tabularnewline
  & Medicare & 26.37\tabularnewline
  & No charge & 0.58\tabularnewline
  & Other & 3.07\tabularnewline
  & Private insurance & 46.9\tabularnewline
  & Self-pay & 4.86\tabularnewline
  & Unknown & 1.93\tabularnewline
  & Worker's compensation & 5.17\tabularnewline
  UrbanCategory & &\tabularnewline
  & Large central metro & 24.6\tabularnewline
  & Large fringe metro & 14.62\tabularnewline
  & Medium metro & 34.55\tabularnewline
  & Micropolitan/noncore (nonmetro) & 15.98\tabularnewline
  & Missing data & 0\tabularnewline
  & Small metro & 10.25\tabularnewline
  PercentPoverty & &\tabularnewline
  & Missing data & 0\tabularnewline
  & Quartile 1 (Less than 5.00 percent) & 18.16\tabularnewline
  & Quartile 2 (5.00-9.99 percent) & 30.24\tabularnewline
  & Quartile 3 (10.00-19.99 percent) & 39.22\tabularnewline
  & Quartile 4 (20.00 percent or more) & 12.38\tabularnewline
  \bottomrule
  \end{longtable}
  
  \doublespacing
  Below are the summary statistics for an example sample:
  
  \begin{Shaded}
  \begin{Highlighting}[]
  \KeywordTok{summarize_each}\NormalTok{(sample1, }\KeywordTok{funs}\NormalTok{(mean))}
  \end{Highlighting}
  \end{Shaded}
  
  \subsubsection{insert table here}\label{insert-table-here}
  
  Within just this one random sample, we can observe that urgent care
  visitor are mostly white, urban and not wealthy. There is an almost even
  mix of those who have private insurance and those who dont, only about
  7\% pay out of pocket, and there seems to be an even mix of sexes.
  
  Four other methodological decisions were made by following either
  statistical convention or similar previous anlayses. All variables were
  created as dummy variables, and the distances between these were
  standardized on a scale from 0 to 1, so as to prevent any skewdness
  which might result. Second, I chose to use the measure of distance known
  as the Jaccard method, which is specifically created to measure the
  distance between 0 to 1 scaled variables. It also has the unique feature
  of not including as significant pairs which both have 0 for a parameter.
  This is substantively important since though two visits may both have
  0's for Private Insurance for example, the fact that they both don't
  have private insurance is not enough to consider them theoretically
  simliar by negation: one may be on medicare while the other may be
  uninsured. Third, for the actual grouping themselves, I have chosen the
  standard Ward's method, which attempts to minimize the variance within
  groups and thus maximizes the homogenaity within groups. Fourth, in
  keeping with similar exploratory anlyses, I have limited the clusters to
  a theoretically interesting number while keeping a managable
  representation of reality.
  
  This can be thought of as us having a set of
  \[X_{1}, X_{2}, X_{3}, ..., X_{n}\] observations.
  
  First, we have standardized all the variables we used on a scale from 0
  to 1, to prevent the sort of skewed analysis that might result if some
  variables with a broad range of absolute values were allowed to dominate
  the data analysis. Second, we have chosen the classic measure of
  distance known as `squared Euclidean' to evaluate the similarities
  between cases, as it gives more importance to greater distances, and
  thus makes it possible to bring out the differences between countries
  whose profiles still show high degrees of similarity.12 Third, for the
  actual groupings themselves, we have adopted the usual Ward's method,
  which minimizes the variance within groups and thus maximizes their hom-
  ogeneity. Fourth, in keeping with normal practices for exploratory
  analyses
  
  In the supervised learning world, the data in question has an obvious
  response variable, which is tested against a null hypothesis based on
  theory. For this case, we do not know exactly what the outcome of
  interest is for those patients who went to urgent care, rather we are
  interested in who is choosing to go there. For such a data set,
  clustering methods allow us to examine the data in an
  \emph{unsupervised} manner --- mainly in that we let the statistical
  software find the patterns rather than test for patterns at the start.
  
  \section{Hierarchal Cluster Analysis}\label{hierarchal-cluster-analysis}
  
  This method allows for grouping patients that have similar
  characteristics across a set of variables by dividing a set of cases
  into ever more numerous and specific subsets, thus leading to homogenous
  empirical types (Rapkin and Luke, 1993). One of the most powerful
  exploratory aspects of cluster analysis is that you do not need to have
  a response variable in order to better understand your data. For this
  project, this is extremely useful since we initially only know who is
  going to urgent care and who is not, but would like to understand them
  as a group better before drawing comparisions between patients who
  visited a traidtional primary care clinic. Also a plus for cluster
  analysis, such inductive methodologies are based only on quantitative
  similarities among cases, only two factors may be responsible for trends
  in the data: the actual structure of the observed phenomenon and the
  methodological decisions I made concerning choosing the cases and
  variables (inclduing the statistical method used to identify subsets).
  
  Because I am interested in two somewhat distinct aspects of the patients
  of Urgent Care-- both their demographics and their patterns of use --
  The clustering was performed in two batches of parameters which aimed to
  help us understand the two different side to a visit to urgent care.
  Variables for \emph{Age, Sex, Race, Urban Type, \% Neighborhood Poverty,
  \% Neighborhood college degree attainment, and Pay Type}, what I will
  refer to as the \emph{demographic variables} from this point forward
  were first analyzed for subgroups. Secondly, some of the same variables
  were again analyzed with the behavior parameters of \emph{Injury
  related, Primary Caregiver, Seen Before?, Past Visits, Major Reason, and
  the day of the week}, what I will refer to as the \emph{behavior
  parameters}.
  
  While clustering methods are incresingly being used to generate
  scientific hypotheses, this analysis rather aims to apply clustering as
  a method of retroduction on a phenomenon that is currently vastly under
  studied. The methodological decisions of this study are divided into
  three areas: (1) the determination of the number of clusters, (2) the
  examination of how the clusters differ in terms of multiple parameters,
  and (3) examining the clusters in light of the theoretical hypotheses
  posed in the introduction.
  
  \subsection{Determining the number of
  Clusters}\label{determining-the-number-of-clusters}
  
  \emph{soon to be in the appendix}
  
  Because urgent care centers have been greatly ignored by sociologists
  studying medical practices, there was little theoretical guidance in
  selecting a likely number of subgroups for the analysis. Similarly,
  because I am interested in understanding how an unsupervised analysis of
  patient data will reveal trends, it was particuarly important to the
  analysis that the number of clusters were both methematically acheivable
  and substantivly small enough for analysis.
  
  To accomplish this, I began with hierarchal clustering of the random
  samples chosen from the data. Using a mixed-methods tool to calculate
  the distance matrix between the various observations, I placed each
  point into an althgorithm which subsequently minimized the variance
  between clusters. Figure 1 shows the banner for the initial
  agglomerative cluster methods for the behavioral variables. .
  
  The white lines extending to the right represent clusters which differ
  from each other. After running the hierarchal, bottom up method for a
  number of trials, the agglomerative coefficient was always between .73
  and .8, indicating that as the height for which the clusters should stop
  combining. Again, Figure 1 demonstrates that at that height, 8 clusters
  are have clearly seperated.
  
  \chapter{Analysis}\label{analysis}
  
  \onehalfspacing
  
  \section{not rly sure what this section is anymore
  \ldots{}.}\label{not-rly-sure-what-this-section-is-anymore-.}
  
  A quick summary of the data shows that by and large most of the
  observations continue to choose traditional means of obcuring primary
  care. Great quote: For our purposes, we do not presume to go quite this
  far, but we are applying this method as a form of retroduction, using
  observed evidence to create a research hypothesis that accounts for the
  observed facts \emph{Sayer, 1992}.
  
  \section{Identifying Typologies with Cluster
  Analysis}\label{identifying-typologies-with-cluster-analysis}
  
  While clustering methods are incresingly being used to generate
  scientific hypotheses, this analysis rather aims to apply clustering as
  a method of retroduction on a phenomenon that is currently vastly under
  studied. The results of this study are divided into three areas: (1) the
  determination of the number of clusters, (2) the examination of how the
  clusters differ in terms of multiple parameters, and (3) examining the
  clusters in light of the theoretical hypotheses posed in the
  introduction.
  
  Because I am interested in two somewhat distinct aspects of the patients
  of Urgent Care-- both their demographics and their patterns of use --
  The clustering was performed in two batches of parameters which aimed to
  help us understand the two different side to a visit to urgent care.
  Variables for \emph{Age, Sex, Race, Urban Type, \% Neighborhood Poverty,
  \% Neighborhood college degree attainment, and Pay Type}, what I will
  refer to as the \emph{demographic variables} from this point forward
  were first analyzed for subgroups. Secondly, some of the same variables
  were again analyzed with the behavior parameters of \emph{Injury
  related, Primary Caregiver, Seen Before?, Past Visits, Major Reason, and
  the day of the week}, what I will refer to as the \emph{behavior
  parameters}.
  
  \subsection{Determining the number of
  Clusters}\label{determining-the-number-of-clusters-1}
  
  \emph{soon to be in the appendix}
  
  Because urgent care centers have been greatly ignored by sociologists
  studying medical practices, there was little theoretical guidance in
  selecting a likely number of subgroups for the analysis. Similarly,
  because I am interested in understanding how an unsupervised analysis of
  patient data will reveal trends, it was particuarly important to the
  analysis that the number of clusters were both methematically acheivable
  and substantivly small enough for analysis.
  
  To accomplish this, I began with hierarchal clustering of the random
  samples chosen from the data. Using a mixed-methods tool to calculate
  the distance matrix between the various observations, I placed each
  point into an althgorithm which subsequently minimized the variance
  between clusters. \textbf{Firuge} shows the banner for the initial
  agglomerative cluster methods for the behavioral variables. .
  
  The white lines extending to the right represent clusters which differ
  from each other. After running the hierarchal, bottom up method for a
  number of trials, the agglomerative coefficient was always between .73
  and .8, indicating that as the height for which the clusters should stop
  combining. Again, \textbf{fig} demonstrates that at that height, 8
  clusters are have clearly seperated.
  
  \subsection{Differences in Demographic and Behavioral Factors Between
  Clusters}\label{differences-in-demographic-and-behavioral-factors-between-clusters}
  
  With cluster assignments established, I then analyzed how the identified
  groups differed from each other in terms of critical demographic
  chracteristics and behaviors that are described in the literature. Most
  importantly---in order to better understand the characteristics of
  individuals choosing to go to urgent care we must understand which
  chracteristics or bahaviors most distinguished the clusters in the
  anlaysis.
  
  Beginning with demographic trends, it is immediatly clear that there are
  large homogenous groups within the selection of patients who utilize
  urgent care. Of the 8 clusters, 7 are predominantly or exclusively
  white.
  
  ** old table location **
  
  \section{With commentary}\label{with-commentary}
  
  The exploratory analysis to follow was motivated by the current lack of
  statistical analysis and social theory surround urgent care centers,
  despite their rapid progression as an acceptable replacement for primary
  care (erhm). The exploratory nature of the findings will allow those
  intent on locating urgent care centers within the larger contextual
  framework of the American health care system to draw on the revealed
  typologies of urgent care seekers in order to better understand the
  industry's rapid growth. {[}UNCECESSARY!{]}
  
  Because I am particularly interested in two somewhat individual facets
  of the patients of urgent care---both their demographics and their
  patterns of use---the clustering was performed in two batches of
  parameters which aimed to help develop an understanding of the two
  components which are involved in a visit to urgent care{[}WHOA TORTURNED
  SENTENCE{]} . Variables for Age, Sex, Race, Urban Type, \% Neighborhood
  Poverty, \% Neighborhood college degree attainment, and Pay Type, which
  I will refer to as the demographic variables from this point forward,
  were first analyzed for subgroups.
  
  Subsequently, some of the same variables were again analyzed with the
  parameters of Injury related, Primary Caregiver, Seen Before, Past
  Visits, Major Reason, and the day of the week, which I will refer to as
  the behavior parameters. \emph{{[}FYI, you should have a section of your
  methods discussion dedicated to explaining each of these variables.{]}}
  
  The analysis of this study is divided into three areas: (1) the
  determination of the number of clusters, (2) the examination of how the
  clusters differ in terms of multiple parameters, and (3) examining the
  clusters in light of the theoretical hypotheses posed in the
  introduction.
  
  \emph{Determining the number of Clusters}\\
  \textbf{{[}Most of this can go in an appendix{]}}
  
  Because urgent care centers have been significantly overlooked by
  sociologists studying medical practices, there was little theoretical
  guidance in selecting a likely number of subgroups for the cluster
  analysis, often the first step in such examinations. Similarly, because
  I am interested in understanding how an unsupervised analysis of patient
  data will reveal trends, rather than imposing them onto the data, it was
  particularly important to the investigation that the number of clusters
  be both mathematically achievable and substantively small enough for
  analysis while still remaining unsupervised. The general rule of thumb
  for cluster analysis suggested by Mardia, Kent and Bibbby (1997)
  stipulates that the number of clusters k is approximately the square
  root of n / 2. However, in the case of our comparably small sample of
  urgent care visitors recorded in the NAMCS data set, such a rule would
  indicate at least 30 clusters---clearly not a workable number to make
  sense of patterns in urgent care patients and their use.
  
  Instead, to estimate a k which would produce heterogeneous groups while
  still remaining theoretically enlightening, I began with non-guided
  hierarchal clustering of samples of 200 visits to urgent care centers
  randomly chosen from the data. Using the Gower methodology of finding
  the distance between dissimilar variable types, I calculated the
  distance matrix between the various observations for each sample. These
  distance measures were then used as the input to a hierarchal clustering
  algorithm which attempted to minimize the distance between observations
  within the same groups while maximizing the distance between them.
  Figure 1 shows the `banner' for the initial agglomerative cluster
  methods for the behavioral variables, and can be understood as a
  graphical representation of the points at which a cluster breaks away
  from the pack of observations. The white lines extending to the right
  represent clusters at their separation point, where larger red areas
  between white lines indicate stronger outside group variance.
  
  \emph{Figure 1. Agglomerative Cluster Banner}
  
  After running the hierarchal method for a number of trials (on-going),
  the agglomerative coefficient for the behavioral parameters remained
  between .73 and .8 (so-far), indicating that level as the most-likely
  optimal height for which the clusters should stop combining in order to
  maximize their heterogeneity and avoid overfitting the groups. For the
  demographic variables, the coefficient ranged between .8 and .85, with
  an average of 12 clusters at those heights. Again, Figure 1 demonstrates
  graphically that at roughly height .75, 8 clusters have clearly
  separated and to go further left would dramatically decrease the
  differences between groups. For the demographic variables, the
  coefficient ranged between .8 and .85, with an average of 12 clusters at
  those heights. Accordingly, the demographic set of parameters and the
  behavioral set of parameters were assigned 8 and 12 clusters
  respectively, which were then used to sort the samples into homogenous
  groups.
  
  \subsection{Differences in Demographic and Behavioral Factors Between
  Clusters}\label{differences-in-demographic-and-behavioral-factors-between-clusters-1}
  
  In order to better understand the characteristics of individuals
  choosing to go to urgent care, we must first understand which
  characteristics or behaviors most distinguished the clusters in the
  cluster analysis, and to do this I used the k-modes clustering algorithm
  to situation observations into similar groups of patients. With cluster
  assignments established by the hierarchal agglomerative methods above, I
  then analyzed how the identified groups or 8 and 12 differed from each
  other in terms of critical demographic characteristics and visit
  behaviors that were described in the literature review.
  
  Beginning with demographic trends, the cluster analysis makes
  immediately clear that there are distinct homogenous groups within the
  selection of patients who utilize urgent care. Race in particular is
  immediately noticeable for its lack of variety: of the 12 clusters, 9
  are predominantly or exclusively white, an unsurprising trend when one
  looks at the summary statistics (of the 3863 urgent care visits included
  in the data, 87\% (3352) of those were white individuals). Table 1 shows
  the modes of each cluster for an illustrative sample of patients, and it
  is immediately clear that the majority of urgent care goers are white,
  but that within clusters of white patients there is further
  segmentation.
  
  For an example of such segmentation, we can examine the socioeconomic
  indicators for the patients' ZIP codes which were included as
  demographic parameters. Unsurprisingly, these appeared to cluster and
  correlate with both race and income: for both males and females, there
  were consistent clusters of White, 25-44-year-old patients with private
  insurance, and both of these clusters were in the highest quartile of
  percent population with Bachelor's degrees, the second lowest percent
  population under the poverty line (5-10\%), and both visited clinics
  coded as being located in a ``Large central metro'' (Table 1.).
  
  In fact, if one predominant trend becomes clear from the demographic
  clustering, it is that urgent care centers are almost entirely an urban
  affair. Of the clusters found, all but one consisted of visits located
  in medium to large metros, indicating that while other variation between
  patients exists, most urgent care consumers are located in larger
  cities. And if its singularity wasn't enough, the cluster of visits
  which did occur in a rural setting also differed from most other
  clusters in the other parameters as well. As can be seen in the first
  row of Table 1., the rural (``Micropolitan/noncore'') cluster of visits
  had one of the highest levels of localized poverty, one of the lowest
  levels of educational attainment, and was one of only two clusters where
  the majority of visits were paid with Medicaid.
  
  Also informative are the clusters' most occurring payment types, which
  complicate some of the theoretical notions of urgent care described in
  Chapter 1. While the largest clusters were consistently comprised of
  private insurance payers, Medicare without a doubt plays a role in
  getting people to urgent care centers. Medicaid had the lowest numbers
  of visits across all trials, averaging around 12 \% of the observations.
  These clusters are crucial to understanding the decision process behind
  choosing a health care provider, while simultaneously raising the
  question of the relationship between the surprising number of elderly
  patients on Medicare seeking treatment at an urgent care centers.{[}THIS
  PARAGRAPH SHOULD OFFER A CLEARER STORY -- AND CLARIFY THE DISTINCTION
  BETWEEN MEDICARE AND MEDICAID{]}
  
  If the demographic clusters illuminate the subgroup characteristics of
  urgent care seekers with socioeconomic status in mind, the behavioral
  parameters included in the second wave of cluster analyses allow us to
  go further in examining how urgent care is utilized by different groups.
  {[}READ THAT SENTENCE AND THINK OF YOUR POOR MOTHER.{]} It should be
  noted that though the analyses were performed in two waves, the clusters
  should be examined with each other in mind, and I have included some of
  the key demographic variables in with the behavioral variables to that
  end. {[}OH YEAH, THIS IS GOING IN THE APPENDIX{]}
  
  \emph{Figure 2. Color coded clusters.}
  
  Figure 1 above shows the typical spread of the behavioral parameters'
  clusters, and is useful for keeping the proportions of clusters in mind.
  By far the largest and most homogenous clusters isolated consist of
  white, privately insured patients in medium to large urban environments.
  These trends match those found in the demographic clusters, and we can
  learn even more about this group by examining the cluster segmentation
  that occurred across their behavioral parameters in the second wave. Of
  the two clusters which consist almost entirely of white, 25-44-year-old
  men and women (yellow and light pink in Figure 1 above), both were
  almost entirely classed as ``established patients'', with at least one
  past visit. Also theoretically interesting, neither cluster had
  observations whose reason for visiting was ``injury related'' and
  neither had recorded visits on weekends. Cluster one in particular (Row
  1, Table 2), consisted of 25-44-year-old white females with private
  insurance, an established history of visiting urgent care, and a reason
  for their current visit coded as a ``chronic, routine problem''.
  {[}SUGGESTION: LOW-TECH TABLE TO ILLUSTRATE YOUR CLUSTERS. COLUMNS FOR
  URBAN/RURAL, RACE, AND OTHER KEY VARIABLES.{]}
  
  In fact, only 32 visits in the example model were coded as new patients
  (Figure 2, purple and orange), and examining these two clusters reveals
  what I referred to earlier as the traditional patient characteristics
  ascribed to urgent care visits. These two cluster show what many would
  expect, both consisted of white, privately insured, new patients, and
  both clusters contain the only observations which occurred on a weekend
  or were coded as injury related. These characteristics are in line with
  what many hold as the purpose of urgent care centers, but they remain a
  small fraction of the recorded number of visits. Completely
  contradicting this idea, the other seven of the nine clusters identified
  have high levels of established patients with chronic or routine
  problems, with ages ranging for 15 to over 75.
  
  As for the Medicare question, if the clusters which consist of primarily
  elderly patients are examined---rows 5 and 9 in Table 2 (Figure 2,
  magenta and blue)---we can begin to understand a little better the
  relatively large percentage of what many would consider non-emergent
  care utilizers and their relationship to urgent care. Cluster 5 consists
  entirely of white patients between 65 and 74, whose visits were recorded
  as routine and who have a minimum of 3 past visits at the same center.
  Cluster nine on the other hand has a slightly older age range of above
  75, and visits which were all considered new problems. Also
  distinguishing, cluster nine is the only cluster with a majority of
  rural observations. When compared, it appears that there are two
  typologies of elderly urgent care seekers, the first of which may be
  using urgent in much the same ways as the younger and larger clusters of
  privately insured urbanites, while the second seems to rely on urgent
  care much more as an actually resource for urgent problems.
  
  \subsection{What do these mean?}\label{what-do-these-mean}
  
  \begin{verbatim}
  The question remains how the observed trends align with the literature that assess the ways Americans are accessing healthcare in an increasingly changing market. From the exploratory analysis outlined above, it is possible to use the findings to test the hypotheses outlined in Chapter 1. In doing so, we we compare our set of urgent care visitors to the larger set of observations in the NAMCS that chose traditional means of accessing primary care, testing the parameters which appeared to cluster together within urgent care seekers as possible predictors for urgent care as an outcome (actually, is this something we want to do? I’ve somewhat done this already way back when and most are significant, esp. white, urban, etc. but it might be interesting to take something like chronic routine problem and see what coefficient I get?). 
  \end{verbatim}
  
  (working on making aggregated/pretty/readable versions of these)


  % Index?

\end{document}

